\chapter{Introduction}

%\section{\label{sec:1}Introduction}

\section{Motivation}

Any computation, communication, or information processing task relies on the manipulation of physical systems that serve as information carriers.  As a result, the theoretical underpinning of such tasks is ultimately a physical theory that described the properties of the physical systems used as information carriers.  Classical information theory uses physical systems whose degrees of freedom are distinguishable, at least in principle, such as coins, and voltages pulses. Quantum information theory on the other hand uses physical systems whose degrees of freedom are indistinguishable, such as the spin of an electron, or the polarization of a photon.  Somewhat counter-intuitively, this property of quantum mechanical systems allows for improvements to certain information processing tasks, that classical information theory is not able to achieve.  Examples of such tasks are super-dense coding~\cite{BW92}, establishing a secure secret key~\cite{BB84}, and exponential speed ups to some important algorithmic processes, such as factoring~\cite{S99} and unstructured search~\cite{G96}. 

This thesis focuses mainly on communication protocols that utilize quantum mechanicals systems.  In a communication protocol several parties exchange information, classical or quantum, encoded in the degrees of freedom of some physical system. The sender (Alice) encodes a message by preparing the degrees of freedom of a system,  parametrized by a set of real numbers, in a definite state that represents the message she wishes to relay, and sends this system through a communication channel to the receiver (Bob).  The latter then decodes the message by performing a measurement, inferring the state of the degrees of freedom of the system, whose outcome, either directly or indirectly, represents the receivers translation of the message.  In an ideal communication scenario the encoded and decoded messages are the same.  Unfortunately, real world communication of information is not ideal. 

The assumptions behind an ideal communication scenario are: (i) Alice and Bob have unlimited control over the physical systems, i.e.~they can prepare and measure the degrees of freedom of the system with infinite precision,(ii) the communication channel is inert, and (iii) Alice and Bob agree on how the states of the physical system are defined. For example, if Alice and Bob use a coin to communicate classical information, then they must agree on which side of the coin they call heads, and which they call tails. Condition (iii) above amounts to the observation, often made implicit in communication protocols, that Alice and Bob must share a common frame of reference for the degrees of freedom of the system under consideration. The focus of this thesis is communication of information, both classical and quantum, using quantum mechanical systems between several communication parties who lack a shared frame of reference.  

To illustrate the difficulties that arise in a communication scenario where parties lack a shared frame of reference, consider the following classical communication scenario.  Alice wants to arrange a date with Bob.  Alice holds a passive cavity resonator with a capacitative membrane attached to a quarter-wavelength antenna.  The cavity resonator is made active by a radio signal of a particular frequency $\nu$.  After activating the device, the sound waves produced by Alice's vocal chords modulate the radio signal which is in turn transmitted by the antenna~\footnote{This particular device is due to Theremin, and was used in the Great Seal bug to eavesdrop on the American ambassador to Moscow. (see \url{http://en.wikipedia.org/wiki/Thing_(listening_device)} for a fascinating piece of history)}. The process of representing Alice's message as a modulated radio signal is the \defn{encoding} of the message.  The \defn{communication channel} between Alice and Bob is the atmosphere through which the modulated radio signal travels.  Bob \defn{decodes} the message by homodyning the modulated radio signal with a radio wave of frequency $\nu$ and feeding the result to a speaker. 

In order for this communication protocol to work it is important that Alice and Bob share a synchronous clock, i.e.~a physical system whose period of oscillation is $T$.  If Alice's and Bob's clocks are asynchronous then they do not define $\nu$ in the same way, and Bob's translation of the message will be different from Alice's. No matter what physical systems are used as information carriers in a communication protocol, whether it is  electromagnetic waves, or the direction of an electron's spin, no meaningful information can be reliably decoded if the sender and receiver lack a shared frame of reference for the physical systems in question.  If the relationship between Alice's and Bob's clock rates is known in the example above, then either party can prepare systems, or perform operations on systems, relative to the other parties clock.  Not knowing the relationship between their clock rates, neither party can prepare systems, or perform operations on systems, in the other parties frame of reference. Thus, lacking a shared frame of reference imposes restrictions both on the preparation of physical systems, and on the operations a party can perform relative to the other parties frame of reference.  As I will discuss below, and prove in Sec.~\ref{sec:2}, if the information carriers are quantum mechanical systems, the restrictions imposed by the lack of a shared frame of reference are operationally equivalent to a super-selection rule (SSR).  

Given that two or more parties lack a shared frame of reference, what physical systems best serve the purpose of encoding information about a parties frame of reference, and how is this ability of physical systems best quantified?  Does there exist an efficient way of communicating information even in the absence of a shared frame of reference?  These are the two main questions that are addressed in this thesis.  Particularly, I will introduce a protocol that allows two parties to learn the relationship between their respective frames of reference, as well as a protocol for efficiently communicating information in the absence of a shared frame of reference.  Before outlining the significance of my contributions (Sec.~\ref{Goals}), I provide an overview of the main concepts as well as key results associated with the problem of communication without a shared frame of reference.  These fall under three categories: the relationship between the lack of a shared frame of reference and a SSR, and the impact of the latter on quantum information processing tasks (Sec.~\ref{SSR}), protocols for aligning reference frames using quantum mechanical systems (Sec.~\ref{RF}), and protocols for communicating both classical and quantum information that obviate the need for aligned frames of reference (Sec.~\ref{DFS}).


\section{Background}
\subsection{\label{SSR}Super-selection rules} 
Super-selection rules (SSRs), first introduced by Wick, Wightman, and Wigner (WWW52),  were originally posited as axiomatic restrictions to quantum theory~\cite{WWW52}. By the time  WWW52 wrote their paper particles had been observed in coherent superpositions of position eigenstates, linear and angular momentum eigenstates, but no particles had been observed in a coherent superposition of charge eigenstates, or in superpositions of parity eigenstates. WWW52 postulated that the reason why superpositions of quantities like charge and parity are not observed is due to a fundamental restriction on the observable operators for such quantities, forbidding superpositions of these quantities from ever been observed.
  
The customary assumption in quantum theory is that every Hermitian operator corresponds to a measurable quantity and vice versa~\cite{vN55}.  WWW52 define SSRs as a restriction on the set of observables, $\mathcal{O}$,  such that the latter is a strict subset of all Hermitian operators.  Under a SSR, associated with some conserved quantity $Q$, a party cannot observe superpositions of different eigenstates of $Q$.  Note that SSRs are different from selection rules, which arise when the evolution of a quantum system obeys some underlying symmetry.

If we are to take the argument of WWW52 at face value, then two kinds of quantities exist in nature.  Quantities like momentum, angular momentum, and energy for which no SSRs apply, and quantities like parity and charge for which SSRs do apply.  This dichotomy between conserved quantities was challenged by Aharonov and Susskind (AS67) a decade later~\cite{AS67}.  In particular, AS67 proposed a thought experiment where coherent superpositions of different charge eigenstates could be observed.  Furthermore, they showed how, under particular conditions, SSRs can hold even for quantities like angular momentum.  The following example illustrates the argument.

Consider an electron prepared in a definite spin-up state along the $z$-axis relative to an absolute frame of reference (such as say the fixed stars), and denote this state as $\ket{0}_{\vec{z}}$.  At some time later, the electron passes through a magnetic field pointing along the $x$-axis (See Fig.~\ref{1}) which rotates the spin of the electron  clockwise about this axis by $\pi/2$. 
\begin{figure}[htb]
\centering
\subfloat{\includegraphics[keepaspectratio, width=5cm]{bloch1.eps}}
\qquad
\subfloat{\includegraphics[keepaspectratio, width=5cm]{bloch_2.eps}}
\caption{A spin-1/2 electron originally pointing along the $z$-axis gets subsequently rotated clockwise about the $x$-axis by $\pi/2$ by a magnetic field $\vec{B}$.  Observer $O_1$'s direction of his measuring apparatus is uncorrelated with the magnetic field, whereas observer $O_2$'s direction is correlated with the magnetic field by an angle $\phi$.}
\label{fig:1}
\end{figure}
The state of the spin of the electron after it emerges from the magnetic field is given by  
\begin{equation}
R_{\vec{x}}\left(\frac{\pi}{2}\right)\ket{0}_{\vec{z}}=\frac{1}{\sqrt{2}}\left(\ket{0}_{\vec{z}}+i\ket{1}_{\vec{z}}\right)=\ket{0}_{\vec{y}},
\label{1}
\end{equation}
where $\ket{0}_{\vec{y}}$ represent the spin state of the electron being up along the $y$-axis.  Now consider two observers with measurement apparatus (Stern-Gerlach magnets). The first observer's observer's measurement apparatus is not correlated with the magnetic field, so that the relative orientation between his measurement apparatus and the magnetic field can take any value in the interval $[0,2\pi)$ The second observer's measurement apparatus is orientated along direction $\vec{m}$, such that $\vec{m}\cdot\vec{x}=\cos(\phi)$, where $\phi$ is known and fixed, i.e.~the second observer's measurement apparatus is correlated with the magnetic field. If the latter changes direction, then the observer's measurement apparatus changes accordingly leaving the relative orientation of the two fixed.

The likelihood that the second observer detects the spin of the electron to be spin-up along $\vec{m}$ is, according to quantum mechanics, $\cos^2\left(\frac{2\phi-\pi}{4}\right)$, whereas the probability of detecting the electron to be spin-down along $\vec{m}$ is $\sin^2\left(\frac{2\phi-\pi}{4}\right)$. Notice that if this observer always chooses to set his measurement apparatus at $\phi=\frac{\pi}{2}$, i.e.~along $\vec{y}$, then he will always observe the spin of the electron to be spin-up along that direction, and he immediately infers that the electron's spin is in a superposition as in Eq.~\eqref{1}. Now consider the probabilities that the first observer detects the electron to be spin-up or spin-down along the direction $\vec{m}$.  As the first observer's measurement apparatus is uncorrelated with the magnetic field, we need to average over all possible orientations of the magnetic field $\phi\in[0,2\pi)$ which yields   
\begin{equation}
\frac{1}{2\pi}\int_{-\pi}^{\pi} \cos^2\left(\frac{2\phi-\pi}{4}\right)\mathrm{d}\phi=\int_{-\pi}^{\pi} \sin^2\left(\frac{2\phi-\pi}{4}\right)\mathrm{d}\phi=\frac{1}{2}.
\label{2}
\end{equation}
Moreover, this probability is the same for any direction $\vec{m}$ .  Hence, the first observer  concludes that the state of the electron is an incoherent mixture of spin-up and spin-down which is what one would expect if there was a SSR for angular momentum in place.   

AS67 use a similar argument to show that the reason why no superpositions of charge eigenstates are observed is due to the lack of a reference frame for charge.  They then introduce a gedanken experiment where such a reference frame can be constructed. Wick, Wightman, and Wigner (WWW70) however, argued that the construction given by AS69 creates a coherent superposition of charge eigenstates using a reference frame which does not exist in nature~\cite{WWW70}.  WWW70's argument is that SSRs do not exist for quantities like position, linear momentum or angular momentum because there exist natural systems, such as particles, magnets etc, that are in effect coherent superpositions of eigenstates of the conserved quantities in question. For the example of angular momentum given above, the state of the electron after it emerges from the magnetic field, Eq.~\eqref{1}, is described relative to that magnetic field. As the direction of the latter is known, its total angular momentum is undefined, since the two quantities are conjugate. So the second observer is not subject to a SSR since he has access to a system, namely the magnet, which serves as a reference frame. However, nothing prevents the first observer from obtaining a reference frame of his own. He too can place a magnetic field in the $x-y$ plane such that the relative angle between this magnetic field and his measurement apparatus is correlated. Thus, for the case of angular momentum, WWW70 argue that no \defn{axiomatic} SSR is in place, because there exists natural systems that can serve as reference frames for the relevant quantities.  
%As no natural systems exist to serve as reference frames for charge, axiomatic SSRs forWWW70 insist that super-selection rules for such quantities are axiomatic.

The argument by AS69, as well as Mirman who helped elucidate as to how such systems can be constructed~\cite{M69, M79}, is that one cannot elevate conservation laws to SSRs purely on the fact that no natural systems are known to act as appropriate reference frames.  In short, the only difference between conserved quantities like momentum, position and angular momentum, that do not satisfy SSRs, and of quantities like charge, baryon number, and parity, that do is the difficulty of preparing and maintaining the appropriate frames of reference.  It is worth noting that since the SSR debate, a proposal for constructing coherent superpositions of charge eigenstates, involving superconductors, has been proposed~\cite{KW74}, as well as coherent superpositions of eigenstates of atom number in Bose-Einstein condensates~\cite{CD97, HY96,JY96,YRJ97,DBRS06, DRP11}.

SSRs have been hotly debated in the field of quantum information, and more specifically in quantum optics, where significant progress has been made, particularly in the last decade. As early as 1963,  Glauber had shown that the state of the output field of a laser can be described as a coherent superposition of photon number eigenstates~\cite{G63}. Subsequent quantum optical experiments showed that Glauber's prediction was consistent with all quantum optical phenomena observed in the lab, the most important of which is interference, a trademark phenomenon attributed to quantum states exhibiting coherence between various eigenstate of photon number.  

In addition to the experimental results, the coherent state representation had the advantage that it was easy to work with.  All of the above properties had the quantum optics community convinced that the true state of the field outputted by a laser was indeed a coherent state.  In a seminal paper in 1997, M{\o}lmer challenged this notion, and showed that by a different line of argument one could reach the conclusion that what is outputted by a laser is not a coherent superposition of photon number eigenstates, but an incoherent mixture of photon number eigenstates, i.e.~that there is a photon number SSR in place for the optical field of a laser~\cite{M97}.  Furthermore, M{\o}lmer showed, via numerical simulations, that this description of the laser is not at odds with experimental results arguing that two lasers satisfying a photon number SSR can indeed interfere. M{\o}lmer's result was proved analytically by Sanders et.~al.~(SBRK03)~\cite{SBRK03} who showed that all quantum optical observations can be explained equally well if one assumes that the laser field is subject to a photon number SSR. Moreover, the techniques they developed were no more cumbersome to work with than the coherent state representation of Glauber. 

What M{\o}lmer and SBRK03 pointed out is that the density matrix describing the state of the laser field is a mixed state (a mixed state is a density matrix whose rank is strictly greater than one). A well known property for such states is the freedom of ensemble decompositions~\cite{NC00}: two different ensembles of pure states may be described by the same mixed density matrix.  Indeed, the choice of writing down an ensemble for a given density matrix is a matter of convenience. One can then conveniently assume that a laser produces, with some probability $p_\alpha$, the state $\ket{\alpha}$, which is a coherent superposition of photon number eigenstates, or one can assume that the laser produces, with some probability $p_n$, a pulse with a definite number of photons $n$. SBRK03 point out that, based on the empirical results of quantum optical experiments, one cannot claim that one ensemble decomposition is more valid than the other. The result caused a stir in the quantum optics community, which gave rise to the optical coherence controversy, a debate on whether the output field of a laser is coherent or not. 

The resolution of the controversy came a few years later by Bartlett, Rudolph and Spekkens (BRS06)~\cite{BRS06}.  The solution is non other than what AS69 had pin-pointed as the apparent cause of SSRs. Quantum coherence is reference frame dependent.  The argument of BRS06 is that the quantum state of a system contains information not only about the system itself, but also about the relation of the system to other systems external to it.  Thus, whether or not a system is in a coherent superposition depends on whether or not a reference system exists relative to which the system of interest is described.

With this point of view in mind, the optical coherence controversy is very similar to the paradoxes one faces in special relativity.  Consider two observers, Alice and Bob, who are in relative motion to each other and the well-known paradox of the ladder and the barn. Analyzing the situation from each observers reference frame one reaches two seemingly incompatible conclusions.  The paradox arises because the phenomenon is viewed from two different frames of reference.  BRS06 point out that the optical coherence controversy arises because the state of the quantum system is viewed from two different frames of reference, one that is correlated with the system and one that isn't.  The coherent state description implies that there is a clock relative to which the state of the optical laser is defined.  On the other hand, the photon number SSR point of view assumes no clock is available. 

Another controversy in the field of quantum optics had to do with the definition of entanglement under SSRs.  Consider the state $\ket{\psi}=1\/sqrt{2}(\ket{0}_A\ket{1}_B+\ket{1}_A\ket{1}_B)$, where the subscripts refer to Alice's and Bob's labs and the state $\ket{0}$ ($\ket{1}$), indicate a state of no photons and a single photon respectively.  Notice that $\ket{\psi}$ is a state with definite photon number and thus respects a total photon number SSR.  One definition of an entangled state is that it cannot be written as a separable state, $\ket{\phi}_A\otimes\ket{\phi}_B$.  There exist more operational definitions of entanglement;  a state is considered entangled if it can be used to execute a super-dense coding protocol~\cite{BW92}, or a teleportation protocol~\cite{BBCJPW93}, if it can be used to violate a Bell-type inequality~\cite{CHSH69}, or if the state cannot be generated by Alice and Bob with just local operations and classical communication (LOCC).  

It was argued by Tan et.~al.~\cite{TWC91} and Hardy~\cite{H94,H95} that the state $\ket{\psi}$ above is entangled in that it can be used to violate a Bell-type inequality.  As the state involves only a single photon, their conclusion was coined as the non-locality of a single photon. On the other hand Vaidman~\cite{V95} and Greenberger et.~al.~\cite{GHZ95} argued that a single photon cannot violate a Bell-inequality as at least two photons are needed to do so.  Is the state entangled? 

If no photon number SSR is in place, all definitions of entanglement given above are satisfied by the state $\ket{\psi}$. The latter was proved by van Enk~\cite{vE05a}, by using two cavities, containing a single atom, at each of Alice's and Bob's labs and showed that by allowing the single photon state $\ket{\psi}$ to interact with the atoms in the cavities, the respective atoms in Alice's and Bob's labs would end up in a state that is entangled under all definitions given above.  

This same idea was used by Wiseman and Vaccaro (WV03)~\cite{WV03} earlier to provide an operational definition of entanglement under a particle number SSR.  Suppose that Alice and Bob share the single photon state $\ket{\psi}$ given above and in addition are restricted by a total photon number SSR.  WV03, observed that under such a restriction most operational notions of entanglement, such as violating a Bell-inequality, or performing teleportation, require Alice and Bob to perform operations that violate the photon number SSR.  Thus, under a photon number SSR, some of the notions of entanglement used above would quantify the single photon state, $\ket{\psi}$ as entangled, while others would not.  Hence, the very notion of entanglement needs to be modified in the presence of SSR's.  WV03, proceed to define a new operational measure of entanglement under a SSR, as the amount of entanglement that Alice and Bob can generate between their local registers, for which they do share a reference frame, using the state $\ket{\psi}$ and operations for which they do not share a reference frame.  The local registers are precisely the single atom cavities that van Enk used to argue that the single photon state $\ket{\psi}$ was indeed entangled. 

%The barn and ladder are of length $L$, such that when the two are stationary the ladder fits exactly inside the barn.  Now Bob, who is in a rocket car capable of traveling close to the speed of light, has an identical copy of this ladder mounted on top of his vehicle which is moving towards the barn.  According to Bob, the barn is moving towards him at a speed close to the speed of light, so that it suffers length contraction.  Consequently, Bob infers that the ladder does not fit entirely inside the barn.  But according to Alice, who is in the same frame of reference as the barn, it is the ladder that is moving close to the speed of light.  Hence, the ladder suffers length contraction, and fits exactly inside the barn.  In fact Alice notices that there is a non-zero time interval between the point in time where the back end of the ladder enters the barn, and the point in time where the front end of the ladder exits the barn. The two descriptions appear to be at odds, which is what gives rise to the paradox. Both Alice and Bob are equally correct in their description of the phenomenon.  

SSRs that arise due to a party lacking an appropriate frame of reference, are referred as \defn{induced} SSRs, to distinguish them from possible SSRs described by WWW70 as axiomatic~\cite{BRS07}.  Induced SSRs can also arise due to practical limitations.  An example is in NMR quantum computing where quantum states are represented by different atoms on a molecule, and operations are enacted using radio frequencies.  If more than one molecule are present then radio pulses, whose wavelength is much larger than the separation of between molecules, act collectively on all the atoms.  This imposes a constraint on the type of operations that can be performed: all operations must be symmetric, i.e.~permutation invariant.  Hence, in NMR quantum computing a SSR for particle ordering, associated with the symmetric group, $S_N$, is in place.

It follows from the discussion above, that induced SSRs can be alleviated, at least partially, if a party possesses a system that can be used as a reference frame relative to which other systems can be described. Indeed, Bartlett, Rudolph and Spekkens (BRS07)~\cite{BRS07} state that there is no fundamental reason why axiomatic SSRs cannot be equally alleviated, other than the difficulty of preparing and maintaing an appropriate frame of reference.  For some degrees of freedom, such as spin, systems exist, such as magnets, that are so large they can effectively be considered classical frames of reference.  However, if the reference frame itself is quantum mechanical, i.e.~rather than a magnet, one possesses only a handful of spin-1/2 systems, then it is clear that after a certain number of operations where the reference systems are used, the systems capability to act as reference frame will degrade.  A method is required which would quantify how this degradation depends on the size of the token reference in our possession.  

Such a quantification was provided by Bartlett et.~al.~(BRST06)~\cite{BRST06}. Specifically, the authors introduced an operational measure for the quality of a reference frame as the average probability of successfully estimating the state of a quantum system, originally defined relative to a background (classical) reference frame.  They discover that, for the case of a directional and optical phase reference, the quality of a system to act as a reference frame scales quadratically with its size.   For a directional reference frame the size is given by the systems total spin, $J$, and for a phase reference by the total photon number.  More precisely, for a reference frame of size $J$ the state of $J^2$ quantum systems can be inferred with an average probability of $1-\epsilon$.

However, the source of particles that are measured with the bounded frame of reference by BRST06 are unpolarized (the expectation value of the total spin along the $z$-axis of $n$ systems is less than the inverse of the total spin of the reference system).   Poulin and Yard (PY07) discovered that for the more general case where the reference system is used to measure  polarized particles, the longevity of the reference frame scales linearly with its size~\cite{PY07}.  By analyzing the dynamical evolution of the reference frame, PY07 discover that for the case of unpolarized photons studied by BRST06, the reference frame undergoes a random walk whose dynamics are governed by fluctuations of the sources angular momentum, whereas for polarized sources, the dynamics of the reference frame are governed by the bias, or drift, in the sources angular momentum.  The observation that the dynamics of the reference frame undergo a random walk was also shown in~\cite{BRST07}.

The results outlined thus far imply that a frame of reference for a conserved quantity is a \defn{resource}, and consequently the lack of a shared frame of reference gives rise to a \defn{resource} theory.  As mentioned above, under a $S_N$ SSR a party is restricted in performing only permutation invariant operations. Therefore, under an $S_N$-SSR, only symmetrically invariant states can be prepared, and possession of a non-symmetric state constitutes a resource, similar to how a non-local state constitutes a resource under LOCC~\cite{N99}.  
%For example, if Alice and Bob share the two qubit entangled state $\ket{\Psi^-}=1/\sqrt{2}(\ket{0}_A\ket{1}_B-\ket{1}_A\ket{0}_B$, then they can implement the well-known entanglement-assisted teleportation protocol~\cite{BBCJPW93}, to communicate a quantum state $\ket{\psi}$.  Thus, the consumption of one entangled pair of qubits, allows two parties to communicate quantum information. 

To better understand the restrictions imposed by the lack of a shared frame of reference one needs to understand what a frame of reference is.  A reference frame is a physical system, with particular degrees of freedom parametrized by set of real numbers, relative to which the states of other systems can be described.  By particular degrees of freedom I mean that the states of the physical system are different for different values of the parameters.  As an example, consider the position degree of freedom of a particle, parametrized by the vector $\vec{r}$.  The state of the particles position, $\ket{\vec{r}}$ is described relative to another system, whose position, $\vec{r'}$, is fixed.  This other system could be another particle, or simply the position of the observer.  Where we to translate the position of the reference system so that its degree of freedom is $\vec{r''}$,  then the state of the particle would also change.  Notice however, that we cannot use the spin degree of freedom of a system to describe the position of the particle since the spin of a system remains invariant under translation.  The set of all possible states of the reference system corresponds to the set of all possible transformations of the parameters describing its degrees of freedom.   In this thesis I will focus on reference frames whose set of transformations are isomorphic to a symmetry group $G$. 

In any resource theory it is important to identify what states constitute resources, how such resources can be manipulated and quantified, and what type of quantum information processing tasks can be performed with and without the requisite resources. Verstraete and Cirac (VC03) investigated how the concept of non-locality changes under a particle number SSR, and whether such a restriction can give rise to interesting information processing tasks~\cite{VC03}.  VC03 discovered that the notion of non-locality had to be re-defined under a particle number SSR, as there exists separable states (that is states that can be prepared under LOCC) that cannot be prepared locally. Such state were also identified as resources by Rudolph and Sanders in the context of photon number SSRs~\cite{RB01}. Furthermore, VC03 showed that it is no longer true that any two orthogonal states can be perfectly distinguished, and constructed a perfect data hiding protocol under a local particle number SSR.  In a data hiding protocol, classical or quantum information is distributed amongst several parties in such a way that the message can be read if and only if the parties are provided with the means to perform joint measurements. It was shown previously that, in the absence of SSRs, perfect data hiding using quantum mechanical systems was not possible~\cite{TDL01,DLT02}.

VC03 claimed that the set of quantum information tasks that can be performed in the absence of SSRs is a subset of the set of quantum information tasks that can be implemented in the presence of SSRs. Their claim was proved wrong by Kitaev, Mayers, and Preskill (KMP04)~\cite{KMP04}, who proved that the information theoretic security (that is the unconditional security) of any quantum information protocol cannot be enhanced by the presence of SSRs.  In particular, KMP04 argued that in order for a protocol to be unconditionally secure, one must show that the protocol is secure against any cheater who possess unlimited resources.  Under this assumption, then, KMP04 noted that nothing prevents the cheater from possessing a reference system that helps him/her to lift the SSR. In particular, nothing prevents Alice or Bob from possessing a bounded sized reference frame and reading the message in the data hiding protocol of VC03. Consequently, the information theoretic security of protocols subject to SSRs can be no better than those in the standard quantum formalism. 

The analysis of VC03 and KMP04, as well as the non-locality of a single photon debate outlined earlier, shows that under SSRs traditional notions of entanglement, particularly the measures used to quantify entanglement, have to be revised. A first attempt to do so was undertaken by Vollbrecht and Werner (VW01)~\cite{VW01}, where they considered how to compute two measures of entanglement, the relative entropy of entanglement~\cite{VPRK97} and the entanglement of formation~\cite{BDSW96}, for states that possess a particular symmetry, such as rotationally invariant states~\cite{W89}.  A different approach was considered by Schuch, Verstraete, and Cirac (SVC), who used two additive quantities to quantify non-local resources~\cite{SVC04a, SVC04b}. The first additive measure, the entropy of entanglement~\cite{BBPS96}, quantifies the non-local properties of a state in the absence of SSRs, while the second additive quantity, the super-selected induced variance, which SVC introduce, quantifies non-locality in the presence of a particle number SSR. 

The quantification of entanglement was also studied in the presence of an $S_N$-SSR, such as those that occur in NMR quantum computing. Bartlett, and Wiseman (BW03)~\cite{BW03} and Wiseman Bartlett and Vaccaro (WBV03)~\cite{WBV03} showed that, under such restrictions, the amount of entanglement is much less than what one would naively think by not taking into account the $S_N$-SSR.  Their analysis showed that the amount of entanglement per particle is asymptotically zero, a surprising result considering that in the absence of  an $S_N$-SSR, the amount of entanglement per particle is strictly non-zero.

It was later shown that entanglement under both LOCC and a SSR is more analogous to entanglement theory of mixed states under LOCC alone~\cite{BDSW06,JWBVP06}.  In particular, the result of WBV03 that under a $S_N$-SSR, the entanglement per particle is zero is analogous to bound entanglement present in mixed states~\cite{HHH98}.  The latter is a property of mixed states that cannot be locally prepared and whose entanglement cannot be accessed.  Under a SSR associated with a group $G$, it was shown in ~\cite{BDSW06,JWBVP06} that there is a gap between the set of states that can be prepared under a $G$-SSR and the set of states that are distillable under $G$-SSR (a resource state is distillable if it can be obtained from $N$ copies of less resourceful state using only allowable operations) .  The gap is made up of states that are similar to the bound entangled states in LOCC alone.  However, the authors showed that such entanglement can be activated if the parties are provided with a suitable reference state. i.e.~the entanglement is bound by the restrictions imposed by the SSR and can be activated if the parties are provided with the requisite reference frame. Thus, all states under a $G$-SSR are either locally preparable or distillable.   

Just like the resource theory of entanglement, a resource theory of reference frames seeks to quantify in an operational way what states, and to what degree, can the SSR associated with the lack of a reference frame be lifted. That is, a state $\ket{\psi}$ is deemed more resourceful than a state $\ket{\phi}$ if the probability of success for a particular quantum information task is higher using $\ket{\psi}$ instead of $\ket{\phi}$.   One such quantification is the \defn{refbit}, introduced by van Enk~\cite{vE05b}, defined as a product state shared between Alice and Bob where each state is an equal superposition of eigenstates of the super-selected quantity.   Van Enk shows how certain tasks, such as super-dense coding or quantum teleportation, between parties lacking a common phase reference can be improved when the parties share various amounts of refbits.   In particular, van Enk proves the asymptotic recovery of entanglement in the presence of $S_N$-SSR, shown by WV03 to be asymptotically zero, by showing that in the limit of a large number of refbits, the amount of physical entanglement approaches the amount of entanglement present in subspaces of the total Hilbert space, that are spanned by permutation invariant states.  Likewise, van Enk shows that in the limit of large number of refbits, one physical qubit can be used to encode one logical qubit (a logical qubit is one that carries fungible information).  

A more useful measure for quantifying the resourcefulness of a state, which will feature prominently in my thesis, is the\defn{$G$-asymmetry} of states introduced by Vaccaro et.~al.~(VAWJ08)~\cite{VAWJ08}.  The $G$-asymmetry can be understood as the relative entropy distance between a resource state,  and the closest non-resource state.  It was shown in~\cite{VAWJ08} that the $G$-asymmetry is related to the amount of thermodynamical extractable work from a state.  More precisely, VAWJ08 showed that there exists a trade-off between a states ability to to do mechanical work and its ability to act as a frame of reference. In particular, VAWJ08 showed that, given two states one of which is a bounded sized frame of reference, the amount of work that can be extracted using an invariant state for a reference is less than the amount of work that can be extracted with a non-invariant state.  

The $G$-asymmetry was shown to be equal to the relative entropy of frameness for a symmetry group $G$, a quantity analogous to the relative entropy of entanglement in the resource theory of LOCC, by Gour, Marvian, and Spekkens (GMS09)~\cite{GMS09}.  GMS09 also showed that the $G$-asymmetry quantifies the extend to which a quantum state, corresponding to a bounded-sized reference frame, can emulate the full classical frame of reference. Despite the vast similarities of the two resource theories, GMS09 discover one important difference. While the relative entropy of entanglement quantifies the rate of interconversion of resources in a reversible theory of entanglement,  the relative entropy of frameness does not have a similar interpretation.  In this thesis, I show that such an interpretation of the $G$ asymmetry does exist for almost all Abelian groups.

Finally, several measures of frameness, as well as the conditions under which one can transform one resource state to another, either deterministically or stochastically, under particular SSRs were provided by Gour and Spekkens (GS08)~\cite{GS08}.  Specifically, GS08 determine the single copy deterministic and stochastic transformations for three kinds of SSRs; that associated with the lack of a chiral frame of reference ($\mathbb{Z}_2$), a phase reference ($U(1)$), and a Cartesian frame of reference, ($\mathcal{SU}(2)$).  In addition, GS08 provide both deterministic and stochastic frameness monotones, measures of the frameness of a state that do not increase under the set of allowable operations, and consider asymptotic interconversions in all three SRRs above.  They discover that as the strength of the SSR increases ($\mathbb{Z}_2$ being the weakest SSR and $\mathrm{SU}(2)$ the strongest), the possibility of interconverting between resource states of the SSR decreases.  

\subsection{\label{RF}Alignment of Reference Frames}

The take home message of the previous section is that the restrictions imposed on quantum information tasks by induced SSRs can be lifted if the parties involved have access to a requisite frame of reference. As SSRs do not offer any distinct advantage in terms of information tasks that can be performed, and impose severe restrictions on the states and operations a party can have access to, parties can use some of the resources available to lift the SSR. In particular, consider two parties that wish to communicate information encoded in the degrees of freedom of some physical system, but lack a shared frame of reference for the relevant degrees of freedom.  On possibility for the parties is, prior to any communication task, to use some of their resources to establish a shared frame of reference.  Such a task is known as a \defn{reference frame alignment protocol}, or alignment protocol in short. The task in such protocols is for the sender and receiver to estimate the relationship between their respective frames of reference using the minimal amount of resources (by resources here we mean physical systems and measurements). In what follows, I will outline what has been discovered in connection to this problem. 

The first result in relation to the problem of aligning reference frames was given by Peres and Wooters (PW90)~\cite{PW90} who considered the following problem.  Suppose that two spatially separated parties are given instructions on how to prepare a quantum system in one of three possible ways.  The parties are allowed to communicate classically so that they prepare their respective systems in the same quantum state.  The parties then submit their preparations to a third party (Charlie), whose task is to determine which one of the three possible states the two quantum systems have been prepared in. 

If the two parties are allowed to submit a very large number of systems, then it is known that Charlie can determine the exact state of the system with certainty.  But, if the two parties have access to only a finite number of systems, in this case one each, then it is not possible to determine the state of the system with certainty.  In such a scenario, PW90 showed that  more information about the state of the system is gained if Charlie performs a joint measurement on the two systems, as opposed to measuring each system independently.

PW90 quantified Charlie's knowledge of the state by the average information gain. PW90 discovered that for the case of two copies of a system, a joint measurement yields a higher average information gain than a separable measurement.  In addition, PW90 also showed that for two different types of separable measurements, projective measurements and generalized measurements, the latter gives a higher average information gain than the former~\footnote{A projective measurement is given by a set of operators  $P_i$ that satisfy $\sum_i P_i=I$ and $P_iP_j=\delta_{ij}$.  On the other hand, a generalized measurement only satisfies the condition $\sum_i P_i=I$.}. PW90 wondered if their findings regarding measurements are true in general.

The task was undertaken by several physicists in subsequent years.  The first attempt was by Massar and Popescu (MP95) who generalized the problem to the following~\cite{MP95}.  A devise produces a finite number, $N$, of spin-1/2 particles whose spins are parallel and point in some random direction $\vec{n}$ chosen uniformly over the sphere. The task is to estimate the direction $\vec{n}$ of the spins by performing any measurement allowable by quantum mechanics.  MP95 quantified their estimate, $\vec{m}$, of the direction of the spins by the fidelity, $\cos\left(\frac{\theta}{2}\right)$, where $\theta$ is the angle between the estimated direction $\vec{m}$, and the true direction $\vec{n}$.  MP95 showed that the measurement that optimizes the average fidelity is a joint measurement on the $N$ systems, given by a continuously parametrized positive operator valued measure (POVM) (a POVM is a generalized measurement as defined by PW90).  They also showed that no finite sequence of separable measurements can do better, where a sequence of separable measurements is defined as a measurement performed on the first spin whose outcome determines the measurement on the second spin and so on. 

MP95's result is truly remarkable. It is a rigorous statement of the old proverbial rhyme that the whole is more than its parts.  However, their result optimizes the average fidelity and not the average information gain of PW90. In addition, their optimal measurement, while correct in principle, cannot be physically realized.  A physically realizable measurement, i.e.~a POVM with a finite number of parameters, was shown to achieve the optimal average fidelity of MP95 in~\cite{DBE98},  In fact, the authors provided a general procedure for obtaining such measurements both for the case of $\mathrm{SU}(2)$, associated with spin-1/2 systems, and for the case of $U(1)$, associated with photon polarization where the axis of polarization is chosen uniformly at random.  

That the physical measurements of~\cite{DBE98} also optimize the average information gain was shown by Tarrach and Vidal (TV99)~\cite{TV99}.  In particular, TV99 showed that for the case of spin-1/2 systems, whose direction is chosen uniformly at random, the optimal measurements are independent of how one chooses to quantify the estimation.  In addition, TV99 found minimal measurements, i.e.~ measurements with the least number of parameters, that achieve the optimal information gain.  

To see how the above results relate to reference frame alignment let Alice and Bob lack a shared directional frame of reference and who wish to perform reference frame alignment. Alice prepares $N$ spin-1/2 particles pointing in her directional reference frame (say the $z$-axis) and sends them to Bob, whose state of knowledge about the direction of the $N$ systems is that they point along some direction $\vec{n}$.  What measurement must Bob perform in order to gain as much information about Alice's directional frame of reference?  This is exactly the scenario considered by PW90, MP95 and TV99;  Bob must perform a joint measurement on the $N$-systems using the measurement operators obtained by MP95, or TV99.

Surprisingly, it turns out that sending $N$ spin-1/2 systems, whose spins point in the same direction is not the best strategy for Alice and Bob.  Gisin and Popescu (GS99) showed that if Alice can only send two systems to Bob, anti-parallel spins achieve a much higher average fidelity that parallel spins~\cite{GS99}.  This is a remarkably counter-intuitive result, and a uniquely quantum mechanical phenomenon, as in the classical case one can expect that two gyroscopes pointing in the same direction are just as good direction indicators as two gyroscopes pointing in opposite directions.  

GP99 noticed that the orbit of two parallel spins under a general $U\in\mathrm{SU}(2)$, i.e.~the set of states $\{U(g)\otimes U(g)\ket{00}_{\vec{n}}\}$, spans a three-dimensional subspace of the total Hilbert space, $\cH_2^{\otimes 2}$, of two spins, whereas the orbit of two anti-parallel spins, spans the entire $\cH_2^{\otimes 2}$ space.  Thus, the latter is in a sense more distinguishable via measurement than the former, and a joint measurement is required in order to maximize the average fidelity. 

GP99's result of why two anti-parallel spins are better direction indicators than two parallel spins can be rationalized in the context of SSRs.  Lacking a directional frame of reference imposes a SSR for angular momentum on Alice and Bob.  Based on the results in the previous subsection, an appropriate reference frame for such a SSR is a system whose state is a coherent superposition of eigenstates of total angular momentum.  Two anti-parallel spins are described by precisely such a state, whereas two parallel spins are described by a single eigenstate of total angular momentum.

GP99's result was generalized to the case of $N$ spin-1/2 systems by Bagan et.~al.~ (BBBM00)~\cite{BBBM00}. BBBM00 verified that the overall state of $N$ spin-1/2 systems, whose orbit under the action of the group spans the largest possible subspace $\cH\in\cH^{\otimes N}$, is a linear superposition over as many eigenstates of total angular momentum as possible.  Such a state is, in general, an $N$-partite entangled state, that is an eigenstate of $J_{\vec{n}}$, the component of the total angular momentum of $N$ spins pointing along the direction, $\vec{n}$, of Alice's directional frame.  BBBM00 showed that the average fidelity for a protocol utilizing a product state of $N$ parallel spins approaches unity linearly in $1/N$, whereas a protocol utilizing the eigenstate of $J_{\vec{n}}$, approaches unity quadratically in $1/N$. Numerical simulations by Peres and Scudo (PS01)~\cite{PS01a} showed that in fact the $N$-partite entangled state need only be an eigenstate of $J_{\vec{z}}$, a result which was analytically confirmed by Bagan et.~al.~\cite{BBBMT01}.  Both these results are in agreement with what one would expect of a directional frame of reference.  The state of the system must be a linear superposition of eigenstates of total angular momentum.  The more terms in the superposition the better the state is as a direction indicator.  For this to occur, the state must have the smallest, non-negative value for $\vec{J}\cdot\vec{n}$.   

Can alignment protocols with similar efficiencies be constructed for a full Cartesian frame?  The answer to this question provided another big surprise as to the power of quantum mechanics.  Firstly, Peres and Scudo (PS01) showed that a full Cartesian frame of reference can be aligned using a single atom of hydrogen in a Coulomb potential~\cite{PS01b}. Using the energy states of the Hydrogen atom as degrees of freedom, PS01 showed that this can be achieved with average fidelity that approaches unity with the square root of the highest energy state. This time, the states of the Hydrogen atom that optimize the average fidelity are superpositions of eigenstates of both total angular momentum and it's component in the $z$-direction.

At the same time Bagan et.~al.~(BBM01)considered the alignment of a Cartesian frame using a different approach~\cite{BBM01}.  Rather than the energy states of a Hydrogen atom, BBM01 used $N$ spin-1/2 systems, and quantified the success of the estimation using the average error, rather than the average fidelity. Unlike PS01, BBM01 discovered that the optimal state for their protocol is an eigenstate of the component of total angular momentum in the $z$ direction, and that the average error of transmission scales linearly in $1/N$. In addition, BBM01 provide a measurement with finite elements, as opposed to the continuously parametrized POVM of PS01. However, Peres and Scudo were very critical of BBM01's result, claiming that if one where to use $N$-spin-1/2 systems a protocol that achieves a much lower average error exists~\cite{PS02a,PS02b}.  Specifically, one can use half of the spins to transmit the direction of the $z$-axis and the other half to transmit the direction of the $x$-axis. Since protocols for transmitting a direction achieve an average fidelity that scales quadratically with $1/N$, Peres and Scudo suggest that this simpler protocol, which involves Bob performing two separate measurements instead of the one introduced by BBM01, achieves a much lower average error than BBM01. 

The difference between the results of BBM01 and PS01, as well as criticism of the latter about the results of the former, raised two important points.  The first has to do with how the state that best allows Bob to infer the relation between his and Alice's frames of reference depends on the way Alice and Bob quantify the success of their protocol.  The second has to do with the measurement strategy that optimizes the criterion used by Alice and Bob to quantify the success of their protocol.  The latter is quantified by a function that depends only on the true relation between Alice's and Bob's frame of reference and the estimate of this relation obtained from the protocol itself. Denoting such a function as $f(g,g')$, where $g\in G$ denotes the true relation between Alice's and Bob's frames of reference, and $g'\in G$ denotes Bob's estimate of the latter, it was shown by Holevo that for a class of covariant functions, i.e.~functions that satisfy the condition $f(hg,h'g)=f(g,g'),\, \forall h,g,g'\in G$ of which both the average fidelity and average error are particular cases, the optimal measurement is a covariant measurement~\cite{H80}. A covariant measurement consists of a set of measurement operators $\{E_g; g\in G\}$, satisfying $\sum_{g\in G}E_g=I$, which form the orbit of a single, fiducial element, $E_e$, under the action of the group $G$ associated with the frame of reference in question.  With the exception of Peres and Scudo's counter example to BBM01, the measurements for all alignment protocols mentioned above satisfy Holevo's criterion.      

In addition, Massar showed that for the case of two parties lacking a directional frame of reference the state that best allows Bob to infer Alice's reference frame depends on the particular choice of function, $f(g,g')$~\cite{M00}. Massar even constructed a particular covariant function for which the parallel, and not the anti-parallel spins, yielded the optimal strategy.  Massar's result explains why BBM01 and PS01 obtain different states for aligning a Cartesian frame of reference.  However, Peres and Scudo's simpler protocol for aligning a Cartesian reference frame using $N$ spins seems to contradict Holevo's result, as their measurements achieve a higher value for the fidelity than those of BBM01 and are not covariant.  

The resolution to this apparent discrepancy was provided by Chiribella et.~al.~(CDS05)~\cite{CDS05}.  Using the machinery of representation theory, which I shall review in Sec.~\ref{sec:2},  CDS05 derive the set of states and measurements that minimize/maximize any covariant function, $f(g,g')$.  Their optimal measurements depend on the representation $U$ of $G$ under consideration, but their analysis is completely general and allows for the re-derivation of the optimal states and measurements in all cases considered above. The optimal measurements are of the type described by Holevo, with the fiducial element a rank-one operator in a superposition of eigenstates of the conserved quantity in question. The optimal states are also superpositions of eigenstates of the conserved quantity, with the coefficients in the superposition depending on the function, $f(g,g')$ used to quantify the success of the protocol.  Furthermore, Chiribella et.~al.~derive a class of measurements, for a set of states that form the orbit under a particular representation $U$ of a group $G$, that maximize the average likelihood of a correct guess~\cite{CDPS04b}. 

Another proposal for aligning a Cartesian frame of reference was put forth by Ac\'{i}n et.~al.~(AJV01)~\cite{AJV01} that made use of shared prior entanglement between Alice and Bob.  AJV01's approach was to recast the problem between two parties lacking a shared frame of reference, and sharing an ideal quantum channel, to one where the parties share a frame of reference and a noisy quantum channel (see fig.\ref{2}). 
\begin{figure}[htb]
\centering
\[\Qcircuit @C=.5em @R=0em @!R {
&\lstick{\ket{\psi}}\qw&\qw&\qw&\gate{p_g,\,U_g; g\in G}&\qw&\qw&\qw&\meter
}
\]
\caption{Two parties with a common frame of reference, sharing a quantum channel, which, with probability $p_g$, performs the same unitary operations $U_g$ every time.  Here, $\ket{\psi}\in\cH_d^{\otimes N}$, and $U$ forms a unitary representation of the compact, or Lie group $G$ on $\cH_d$.}
\label{fig:2}
\end{figure}
The action of the channel on any input state is described by a black box, that is known to perform some unitary operator $U\in \mathrm{SU}(d)$ which Alice and Bob are initially completely ignorant about. Using the average fidelity to quantify the success of the estimation, AJV01 provide the optimal state of two systems and four systems, which are in an entangled state shared between Alice and Bob, and the optimal measurement that Bob needs to perform in order to determine the operation $U\in \mathrm{SU}(d)$.  AJV01's protocol requires Alice to send her half of the entangled state through the channel to Bob.  For the case of a single transmission of a $d$-dimensional quantum system, AJV01 show that the average fidelity scales as $2/d^2$.  If $d=2$, knowing the exact unitary amounts to knowing the three Euler angles, that parametrize $\mathrm{SU}(2)$.  Hence, estimating the dynamics of a quantum black box is equivalent to Alice and Bob communicating a full Cartesian frame by transmitting part of an entangled state.  This is drastically different to previous results, as it is clear that previous methods cannot use a single $2$-level system to transmit information about a full Cartesian reference frame.

Bagan et.~al.~(BBM04) extended the ideas of AJV01 for the case where Alice and Bob use $2N$ spin-1/2 systems, prepared initially in an entangled state, to transmit a Cartesian frame of reference~\cite{BBM04a}.  In particular, they derive the optimal $2N$ qubit entangled state and optimal covariant measurement that maximizes the average fidelity per axis of the Cartesian frame of reference. They discover that in the limit of large $N$, the average fidelity per axis approaches unity quadratically in $1/N$, as opposed to linearly in $1/N$ using $N$ spins without shared prior entanglement.  This improvement comes at the cost of sharing prior entanglement, whereas the method in~\cite{PS02a,PS02b} achieved the same average fidelity per axis without requiring entanglement.  Later on, both Bagan et.~al.~\cite{BBM04b} and Chiribella et.~al.~\cite{CDPS04a} succeeded in showing that the average fidelity, and average transmission error can be made to approach unity quadratically in $1/N$ without requiring prior shared entanglement.  Their protocols make use of $N$ spin-1/2 systems such that the eigenstates of total angular momentum along a given direction $\vec{n}$, are entangled with a degeneracy index, $\alpha$, which is invariant under the action of a general $U\in\mathrm{SU}(2)$.  This degeneracy index keeps track of the total number of ways a given total angular momentum quantum number, $J$, can arise given $N$ spin-1/2 systems (I shall return to this point later on when I discuss decoherence-free subsystems and also in Sec.{sec:2}). 

It should be noted that the case of aligning a reference frame associated with the group $U(1)$, corresponding to optical phase, using a covariant quantifying function $f(g,g')$, had been dealt much earlier  by Holevo~\cite{H80}.  Similar reference frame alignment schemes also appeared for finite non-abelian groups, such as the symmetric group on $N$ objects~\cite{KK04}.  It should also be noted that a multi-round reference frame alignment protocol has been studied by Rudolph and Grover (RG03) for a directional frame of reference~\cite{RG03}. RG03 showed that such a protocol can achieve a worst case fidelity that approaches unity as $\frac{\log_2 N}{N^2}$ using separable states and separable measurements.  The protocol involves a single spin-1/2 particle being sent back and forth $N$ times through an ideal quantum channel.  At every round, the receiving party performs a measurement and passes the system back to the other party who does the same. Such a multi-round protocol has also been used by de Burgh and Bartlett (dBB05) to perform clock synchronization, and was shown to achieve the same precision in synchronizing two clocks as a protocol using a massively entangled state with only one-way communication~\cite{dBB05}.  The results of both RG03 and dBB05 show that an interesting trade-off between entanglement and quantum communication exists. A protocol utilizing entangled states requires fewer transmissions of quantum systems, but requires entanglement which is expensive, while a multi-round communication scenario requires a much larger number of transmissions, but no entanglement.  

\subsection{\label{DFS}Decoherence-free subspaces and reference frame independent communication}

In the previous subsection, it was shown how two parties lacking a shared frame of reference can use some of their resources to establish a shared frame of reference prior to performing any quantum information task.  However, if after some time their reference frames become misaligned, then they would have to repeat the alignment procedure again, consuming more resources.  Furthermore, there exist cases where even reference frame alignment is not possible, such as in NMR quantum computing, as states that can act as reference frames cannot even be constructed as preparation procedures for such states must necessarily respect the $S_N$-SSR.

In such scenarios, an alternative method for performing quantum information processing tasks is required.  Such a method would have to utilize degrees of freedom of physical systems, whose states can be defined independent of a frame of reference.  In the case of NMR quantum computing for example, one can prepare the $N$ qubits in a state that is permutation invariant, by encoding quantum information in the total angular momentum degree of freedom of $N$ spins.  Similarly, communication parties can encode and decode information in degrees of freedom of physical systems that do not require the particular reference frame the parties are lacking. 

As an example consider two parties that lack a shared phase reference, i.e.~that are restricted by a global photon number SSR, and consider the state $\ket{\psi}=\alpha\ket{01}+\beta\ket{10}$, where $\alpha,\beta\in\mathbb{C}$ and $|\alpha|^2+|\beta|^2=1$. The state $\ket{\psi}$ respects the global photon number SSR, for all $\alpha,\beta\in\mathbb{C}$.  Thus, Alice and Bob describe this state in the same way, even though they lack a shared phase reference.  Alice then can communicate one \defn {logical} qubit to Bob using two \defn{physical} qubits, by choosing the parameters $\alpha$ and $\beta$ accordingly.  The subspace spanned by the single photon eigenstates $\{\ket{01},\ket{10}\}$ is known as an \defn{invariant}, or \defn{decoherence-free} subspace (DFS). 

It is instructive to view the lack of a shared frame of reference between two parties as in Fig.~\ref{fig:2}. Alice and Bob have a common frame of reference, but share a\defn{collective noise} channel to transmit their physical systems.  A collective noise channel is a channel that acts the same way on every system transmitted through it.  The action of the channel is described by a set of unitary operators $\{U_g; g\in G\}$ and a probability distribution $p_g$.  Alice then performs a logical encoding, described by the map $\mathcal{C}$ that maps the quantum state she wishes to transmit, $\ket{\psi}\in\cH_d^{\otimes m}$ to a larger space, $\cH_d^{\otimes N}$, known as the code space.  Alice then sends the physical systems that comprise the code space through the channel to Bob, who decodes the quantum state.  Bob's decoding is described by a map, $\cD:\cH_d^{\otimes N}\rightarrow\cH_d^{\otimes m}$ (see  Fig.~\ref{fig:3})
\begin{figure}[htb]
\centering
\[\Qcircuit @C=.5em @R=0em @!R {
&\lstick{\ket{\psi}}\qw&\qw&\qw&\multigate{10}{p_g,\,U_g; g\in G}&\qw&\qw&\qw&\meter\\
%\push{\rule{0em}{2em}}
&\qw&\qw&\qw&\ghost{p_g,\,U_g; g\in G}&\qw&\qw&\qw&\meter\\
&\qw&\qw&\qw&\ghost{p_g,\,U_g; g\in G}&\qw&\qw&\qw&\meter\\
&\qw&\qw&\qw&\ghost{p_g,\,U_g; g\in G}&\qw&\qw&\qw&\meter\\
&\qw&\qw&\qw&\ghost{p_g,\,U_g; g\in G}&\qw&\qw&\qw&\meter\\
&\lstick{\ket{0}^{\otimes (N-m)}}\qw&\qw&\qw&\ghost{p_g,\,U_g; g\in G}&\qw&\qw&\qw&\meter\\
&\qw&\qw&\qw&\ghost{p_g,\,U_g; g\in G}&\qw&\qw&\qw&\meter\\
&\qw&\qw&\qw&\ghost{p_g,\,U_g; g\in G}&\qw&\qw&\qw&\meter\\
&\qw&\qw&\qw&\ghost{p_g,\,U_g; g\in G}&\qw&\qw&\qw&\meter\\
&\qw&\qw&\qw&\ghost{p_g,\,U_g; g\in G}&\qw&\qw&\qw&\meter\\
&\qw&\qw&\qw&\ghost{p_g,\,U_g; g\in G}&\qw&\qw&\qw&\meter
}
\]
\caption{Encoding and decoding of quantum information in a reference frame independent protocol.  Alice uses $(N-m)$, $d$-dimensional systems, initially in the state $\ket{0}^{\otimes (N-m)}$ and performs the logical encoding of the state $\ket{\psi}\in\cH_d^{\otimes m}$, using the map $\mathcal{C}:\cH_d^{\otimes m}\rightarrow\cH_d^{\otimes N}$.  The channel acts collectively on all $N$ systems with the same operation $U_g$, with some probability $p_g$.  Bob decodes the message, by performing a decoding operation $\cD:\cH_d^{\otimes N}\rightarrow\cH_d^{\otimes m}$ and recovers the state $\ket{\psi}$. }
\label{fig:3}
\end{figure}
 
For such channels, it was shown by Zanardi and Rasetti (ZR97)~\cite{ZR97} that quantum codes can be constructed which are immune to the noise of the channel.  ZR97 realized that if a party had $N$ quantum systems of dimension $d$ at his/her disposal, and wished to use such systems to transmit quantum information through a noisy quantum channel, there exist subspaces of the total Hilbert space $\cH_d^{\otimes N}$ that were not affected by the noise of the channel. Quantum information encoded into these subspaces would remain coherent even after transmission.  Due to the ability of these subspaces to protect quantum information from external noise, they are referred to as decoherence-free subspaces, or DFS.  

The results of ZR97 spurred a huge amount of interest in the quantum information community, as decoherence is a huge nuisance in quantum information.  Lidar, Chuang, and Whaley (LCW98)~\cite{LCW98} added to the hype of DFS by establishing necessary and sufficient conditions for their existence and also demonstrating that universal quantum computation can be performed within a DFS.  A set of logical gates is called \defn{universal} if any possible computation can be reduced to a sequence involving gates from the universal set. 

Subsequent work by Knill, Laflamme, and Viola (KLV00)~\cite{KLV00} gave rise to another type of noiseless encoding using \defn{decoherence-free subsystems}, or \defn{noiseless-subsystems}, NS for short, rather than subspaces. The difference between NS and DFS is rather subtle, and will be explained in more detail in Sec.~\ref{sec:2}.  The crucial point in KLV00 is that for some collective noise channels, such as the one associated with the lack of a Cartesian frame of reference, it is possible to encode logical quantum information using fewer physical qubits, if one utilizes a NS encoding rather than a DFS encoding. For example, one can use three spin-1/2 systems to encode a logical qubit in a NS, but one requires four spin-1/2 systems to encode one logical qubit in a DFS.  

KLV00 showed that NS are equivalent to an error-correcting code with infinite distance, i.e.~an error correcting code that can correct for an arbitrary error.  These NS were studied further by Zanardi, and were shown to arise from the representation of the noise operators of the collective channel~\cite{Z00,Z01}.  Specifically, they are associated with a specific tensor product structure of the total Hilbert space, $\cH^{\otimes N}$, and can be viewed a \defn{virtual} quantum systems, in the sense that their state space does not describe a real physical system, but one that arises from considerations of the symmetries of the noisy channel.  An experimental realization of a NS using three spin-1/2 systems was proposed in~\cite{VFPKLC01} and in ~\cite{YG01} where explicit encoding and decoding circuits where provided.

A unified framework for performing quantum computation on both DFS and NS was provided by Kempe et.~al.~(KBLW01)~\cite{KBLW01}.  In particular, KBLW01 establish necessary and sufficient conditions for the existence of DFS and NS and show that universal quantum computation can be performed on both such constructions. KBLW01 consider specifically collective noise channels associated with the group $\mathbb{Z}_N$, making their results particularly relevant in the context of NMR quantum computing.  In fact, KBLW01 provide the form of the Hamiltonian that gives rise to universal quantum computation in such a setting.  However, the most remarkable result in KBLW01 is that the ratio between the dimension of the logical space  to the dimension of the coding space approaches unity in the asymptotic limit.  In other words, in the limit of a large number of physical systems, such codes achieve the transmission of one logical qubit for every physical qubit.

The result of KBLW01 is profound.  It simply states that if two parties lacking a shared frame of reference have a large number of physical resources at their disposal, the amount of quantum information that they can communicate in the absence of a shared frame of reference is the same as the amount of quantum information they can exchange if they shared a frame of reference.  The only caveat is that, in the absence of a shared frame of reference, Alice and Bob need to implement the maps $\mathcal{C}$ and $\mathcal{D}$ in order to encode and decode information inside a DFS or NS, whereas if the parties shared a frame of reference no such encoding and decoding needs to be applied.

This raises the natural question of how such encoding and decoding maps can be implemented and whether they can be implemented efficiently. The circuit to implement $\mathcal{C}$ and $\mathcal{D}$ was explicitly constructed by Bacon, Chuang and Harrow (BCH)~\cite{BCH06a, BCH06b}.  Specifically, BCH showed that the implementation of the maps $\mathcal{C}$ and $\mathcal{D}$ can be achieved by a specific unitary transformation,  the Schur transform. BCH construct an explicit circuit for this transformation and determine the number of singe, and two-qubit gates that are required in order to implement it.  Their result is that the Schur transform can be implemented efficiently, up to an arbitrary error $\epsilon$, and requires a number of gates that scale as $N\mathrm{poly}(\log_2 N, d, \epsilon)$, where $N$ is the number of physical qubits, and $d$ their dimension.  In this thesis I will provide an alternative protocol for encoding information into a DFS that requires fewer number of gates than BCH's construction and achieves the same asymptotic rate of transmission of KBLW01.

The results about DFS/NS up to this point in the development of the subject were quite general.   While it was proven that quantum information tasks can be done using DFS/NS, how they can be done was still not known.  Bartlett, Rudolph and Spekkens (BRS03)~\cite{BRS03}, showed how DFS/NS can be used to communicate both classical and quantum information between two parties that lack a reference frame associated with the group $\mathrm{SU}(2)$.  In this case, the invariant states are those with definite total angular momentum.  So, for example, if Alice wishes to communicate a single bit to Bob, she can use two spin-1/2 systems prepared in a state of total angular momentum $J=0$, or $J=1$.  To decode the classical bit, Bob measures the total angular momentum of the two spins.  Notice however, that the states prepared by Alice are in general entangled states.

BRS03 proceed to show that if Alice possesses $N$ physical systems, then the amount of classical information she can transmit to Bob approaches $N$, i.e.~in the large $N$ limit Alice can transmit one classical bit per physical system she sends to Bob.  If Alice and Bob wish to send quantum information, then all Alice has to do is encode this information in a DFS or NS. For the case of spins, the noiseless degrees of freedom are associated with the ordering of the spin-1/2 systems, as long as Alice and Bob agree on the order at which systems are sent and arrive, then they can communicate quantum information without sharing a directional frame of reference.  

Using the result of KBLW01, it follows that in the limit of large $N$, Alice can transmit one logical qubit per physical system she sends to Bob.  Notice that Alice and Bob can communicate even entangled quantum states, and as such they can even demonstrate a violation of a Bell-type inequality without having to share a frame of reference for their measurements, a result which was shown explicitly by Cabello~\cite{C03}.  Subsequent work by Boileau et.~al~\cite{BGLPS04} outlined a quantum key distribution protocol, using the polarization states of photons, which utilizes a DFS/NS to guard against the collective noise, caused by birefringence, of optical fibers . Furthermore, it was shown that this protocol was unconditionally secure~\cite{BTBL05}, and an experimental realization of it was shown by~\cite{CZBJYZYLP06}. A separate realization of a DFS/NS quantum key distribution protocol, using time-bin qubits, was shown earlier by~\cite{WASST03}.  

Since the fastest way to communicate between spatially separated parties involves the transmission of light, it is highly desirable to have a complete characterization of how efficiently DFS/NS schemes, utilizing photons as information carriers, can be implemented.  This characterization was done by Ball and Banaszek (BB05)~\cite{BB05}.  Specifically, BB05 consider the transmission of information, encoded in the polarization and phase of photons, via collective depolarizing channels . The latter is the dominant source of decoherence in optical fibers.  BB05 arrive at a recursion formulae for calculating the sizes of DFS/NS given $N$ photons.  In a subsequent paper, Ball and Banaszek extended their result to show that the capacity to transmit quantum information through a collective depolarizing channel requires the use of a hybrid DFS/NS, i.e.~DFS/NS that combine both a polarization and an optical phase encoding~\cite{BB06}. Their scheme requires that the communicating parties share a phase reference but not a directional frame of reference for polarization.  

Finally, the implications of lacking a shared frame of reference for the degrees of freedom of $d$-dimensional quantum systems, and the construction of DFS/NS using such systems where discussed by Byrd~\cite{B06}.  As $d$-dimensional systems are known to exhibit starkly different properties than two-dimensional systems, among which the fact that two $d$-dimensional systems are more entangled than two qubits.  More importantly, the representation theory of $d$-dimensional systems is very different from that of qubits.  As a result, the theory of DFS/NS for $d$-dimensional systems is different from that of qubits.  Byrd shows that for the case of qutrits, i.e.~three-dimensional quantum systems, two different types of states exist and, under a SSR, one cannot transform between these states using operations that respect the SSR.  This is due to the fact that the complete group of symmetries for qutrits contains two inequivalent fundamental irreducible representations. As a result, Byrd shows that one can construct quite different DFS/NS using at least three qutrits required for the smallest possible DFS/NS, that carry a combination of the two fundamental irreducible representations.  

Future work by Bishop and Byrd derived the set of all compatible transformations for DFS/NS constructed by $d$-dimensional systems subject to SSRs~\cite{BB09}.  Bishop and Byrd show that universal quantum computation can be performed on these DFS/NS and construct the Hamiltonian that accomplishes universality for the case of three qutrit DFS/NS.

\section{\label{Goals}My contributions}

In the remainder of this thesis I will introduce two communication protocols. The first is a reference frame alignment protocol that allows two parties to learn the relationship between their respective frames of reference.  It differs substantially from those of Sec.~\ref{sec:RF} in two ways: (i) Alice and Bob have unlimited resources at their disposal, and (ii) Alice and Bob wish to learn, and not estimate the relationship between their reference frames.  As a result the reference frame alignment protocol I propose determines the state and measurement that optimizes the mutual information between the true relationship of Alice's and Bob's frames of reference and the knowledge of the latter about this relationship acquired from measurement.  

The second protocol in this thesis deals with the transmission of quantum information utilizing a DFS.  Specifically, I propose a novel protocol for transmitting quantum data that is efficient (i.e.~achieves the optimal rate of transmission of KBLW01) but requires fewer resources in order to implement the encoding and decoding operations $\mathcal{C}$ and $\mathcal{C}$ than those of BCH~\cite{BCH06a, BCH06b}.  

The impetus of my work, and its connection to the background material outlined in Secs.~(\ref{SSR, RF, DFS}) are  
\begin{enumerate}
\item A reference frame alignment protocol that optimizes the mutual information between the relationship of the sender's and receiver's frame of reference, and the knowledge of the latter about this relationship from measurement.
\item An information theoretic, operational interpretation of the $G$-asymmetry of VAWJ08~\cite{VAWJ08}, which was thus far lacking, as the rate of alignment in a reference frame alignment protocol for the case of a phase reference and a reference frame associated with the finite cyclic group $\mathbb{Z}_M$. 
\item A new measure of frameness, the alignment rate, for the resources in SSR rules associated with the lack of a shared phase reference and a finite cyclic group. 
%\item A uniquely quantum mechanical phenomenon exhibited by the above frameness measure for the case of finite cyclic groups of order greater or equal to four.  In particular, I will show that for two different state $\ket{\psi}$, $\ket{\phi}$ in such resource theories, the rate of alignment is super-additive.
\item A novel quantum communication protocol for noiseless transmission of information through collective channels whose encoding and decoding circuit implementation, in terms of single and two-qubit gates scales, outperforms that of BCH~\cite{BCH06a,BCH06b}
%\item An efficient encoding and decoding circuit for the above protocol, whose implementation in terms of single and two-qubit gates scales linearly with the number of logical qubits and with the number of noise operators of the channel. For the case of finite groups, the number of gates to be implemented at any given time step of the circuit is independent of the number of logical qubits.
\item An extension to the above protocol that deals with collective noise channels associated with continuous groups, specifically $U(1)$, that achieves high fidelity, efficient asymptotic transmission and whose implementation is more efficient than any other DFS communication protocol to date.
\end{enumerate}

 
\section{\label{sec:2}Mathematical Background} 
