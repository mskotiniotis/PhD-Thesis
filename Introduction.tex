\documentclass{article}\bibliographystyle{unsrt}
\usepackage{amsthm,amssymb,amsmath,graphicx,epsfig,color,verbatim,enumerate,dsfont}
\usepackage{epstopdf,rotating}
\DeclareMathAlphabet{\mathscr}{OT1}{pzc}{m}{it}
\newcommand{\op}[1]{\operatorname{#1}}
\newcommand{\cket}[1]{\left|#1\right)}
\newcommand{\ket}[1]{\left|#1\right\rangle}
\newcommand{\cbra}[1]{\left(#1\right|}
\newcommand{\bra}[1]{\left\langle #1\right|}
\newcommand{\bracket}[2]{\left\langle #1|#2\right\rangle}
\newcommand\defn[1]{\textsl{#1}}
\newcommand{\abs}[1]{\left\lvert\tinyspace #1\tinyspace\right\rvert}
\newcommand\ketbra[1]{|#1\rangle\langle#1|}
\newcommand\cH{{\mathscr{H}}}
\newcommand\cM{{\mathcal M}}
\newcommand\cN{{\mathcal N}}
\newcommand\cR{{\mathcal R}}
\newcommand\cG{{\mathcal G}}
\newcommand\cD{{\mathcal D}}
\newcommand\cI{{\mathcal I}}
\newcommand\cB{{\mathcal B}}
\newcommand\cE{{\mathcal E}}
\newcommand\cU{{\mathcal U}}
\newcommand\cV{{\mathcal V}}
\newcommand\cP{{\mathcal P}}
\newcommand\cF{{\mathcal F}}
\newtheorem{lemma}{Lemma}
\newtheorem{proposition}{Proposition}
\newtheorem{theorem}{Theorem}
\newtheorem{corollary}{Corollary}


\begin{document}
\title{PhD Thesis}
\author{Michael Skotiniotis}
\date{\today}
\maketitle


\section{\label{sec:1}Introduction}
A communication protocol involves two or more parties exchanging information, encoded in the states of appropriate physical systems, through a communication link called the channel.  The information the parties wish to exchange may be classical (a string of bits) or quantum (a string of quantum bits or \defn{qubits}) in nature.  In addition, the physical systems which act as information carriers may be classical or quantum in nature, i.e.~governed by classical or quantum mechanical laws.  In an ideal communication scenario, the channel is perfect;  the initial state of the information carriers (the state prior to going through the channel) is the same as the final state of the information carriers (the state after transmission through the channel).  In a real world communication scenario the channel is far from ideal, resulting in transmission errors and corrupt data.  A major goal of both classical and quantum information theory is to devise protocols that are robust against the noise of the channel (error-avoiding codes) or  protocols that allow the communicating parties to account for and correct  transmission errors (error-cerrecting codes).  Furthermore, if some information is known about the nature of the errors incurred from the channel, for example if it is known that the channel acts in the same way on all systems sent through as opposed to independently on each one, suitable protocols whereby parties gain information about the action of the channel can also be implemented prior to communication.

This thesis deals with communication protocols, using quantum mechanical systems as information carriers, between several parties that share a \defn{collective noise channel}, i.e.~a channel that acts the same way on all quantum systems being transmitted.  By exploring the connections between collective noise channels and induced super-selection rules, arising from parties lacking shared frames of reference, I propose a novel protocol for transmitting quantum data that is efficient and requires the least number of resources, both in computational time and space, than any protocol known to date.  In addition, I analyze a reference frame alignment protocol, a task whereby two parties establish a shared frame of reference (or equivalently the task where two parties learn about the collective noise of the channel), as an information theoretic task and provide a protocol that maximizes the amount of accessible information about the relation between the parties respective reference frames.  My approach is different from previous work on the subject and provides an operational interpretation for an important measure of frameness, the $G$-asymmetry. The latter quantifies how suitable a physical system is to act as a frame of reference relative to which the states of other systems can be described. 

The simplest communication scenario one can think of involves two parties, Alice and Bob.  Alice wishes to transmit a message to Bob via a communication link that they share.  For example, Alice may wish to arrange a date with Bob. Conceptually such a scenario is almost trivial, but the physical implementation is rather complex.  One possible way for Alice and Bob to communicate is to use radio frequencies.   Alice holds a passive cavity resonator, with a capacitative membrane attached to a quarter-wavelength antenna.  The cavity resonator is made active by a radio signal of a particular frequency $\nu$.  After activating the device, the sound waves produced by Alice's vocal chords modulate the radio signal, which is in turn transmitted by the antenna. The process of representing Alice's message as a modulated radio signal is known as \defn{encoding} of the message.  The communication link between Alice and Bob is the atmosphere through which the modulated radio signal travels.  Bob \defn{decodes} the message by homodyning the modulated radio signal with a radio wave of frequency $\nu$ and feeding the result to a speaker. 

In order for this communication protocol to work it is important that Alice and Bob share a reference frame for time.  If they do not, then the message that Bob decodes will be incomprehensible. No matter what physical systems are used as information carriers in a communication protocol, whether it is  electromagnetic waves, or the direction of an electron's spin, no meaningful information can be reliably decoded if the sender and receiver lack a shared frame of reference for the physical systems in question.  Moreover, the lack of a shared frame of reference imposes restrictions both on the type of states a party can prepare and on the type of operations a party can perform.  In the case where the information carriers are quantum mechanical systems, the restrictions imposed by the lack of a shared frame of reference are operationally equivalent to a super-selection rule.    
%Thus sharing a common frame of reference is a \defn{resource}. If a shared reference is lacking, it is important to identify what physical systems are resources for such a protocol and to  quantify their resourcefulness.
%
%This thesis is concerned with communication protocols that use quantum mechanical systems, such as the energy levels of atoms, or the optical phase of photons, to physically represent information.  In particular, I will concentrate on physical representations whose description is given relative to an external frame of reference associated with a finite, or compact Lie group, $G$ (In particular I will concentrate on physical representations that transform according to some finite or compact Lie group $G$) where the parties involved in the communication protocol do not share a common reference frame.  My pioneering contributions to the field are:
%
%\begin{enumerate}
% \item A protocol that maximizes the accessible information for aligning two distant raference frames for optical phase
% \item An operational interpretation for the $U(1)$-asymmetry~\cite{Vaccaro}, a realtive entropy measure of resourcefullness defined for all groups.
% \item The capacity to encode and decode information from a highly correlated quantum information source, which extends the known results for the capacity of independent and identically distributed (i.i.d) quantum sources.
% \item  A novell protocol for transmitting quantum information (qubits) whose physical implementation in terms of the number of gates scales linearly with the number of logical qubits being transmitted, and whose rate of transmission in the asymptotic limit is unity. 
%\end{enumerate}
%
%In addition, I have done extensive work on the following two problems
%
%\begin{enumerate}
% \item Optimal encoding and decoding operations for the transmission of messages via decoherence-free subsystems (DFS) where the messages sent belong to a group $H$ that is different from $G$, the group associated with the lack of a shared frame of reference
% \item Optimal encoding and decoding strategies for relative transformations associated with the finite group $S_3$. 
%\end{enumerate}
% 
%I note that while working on the optimal encoding and decoding of messages via DFS it became apparent to both myself and my supervisors that the problem we were addressing had already been solved.  However, I am including my findings in this thesis since as in the process of tackling the problem I have learned a lot about representation theory of groups and about quantum parameter estimation.  In addition, I have obtained analytical results for specific cases, namely when the representations of the message group $H$ and reference frame group $G$ commute, that are rigorous and instructive, and in my opinion elucidate in a simple way what was already known in the field.   Finally, my work on relative parameters contains only partial results. While these are interesting in and of themselves, the completion of this project remains a goal for the future.
 
Super-selection rules, first introduced by Wick, Wightman, and Wigner in 1952,  were originally posited as axiomatic restrictions to quantum theory~\cite{WWW52}.  At the time super-selection rules were introduced particles had been observed in coherent superpositions of position eigenstates, linear and angular momentum eigenstates, but no particles had been observed in a coherent superposition of charge eigenstates, or  in superpositions of parity eigenstates.  Wick, Wightman, and Wigner suggested that the reason that coherent superpositions exist for some quantities and not others could be that for some conserved quantities, like parity and charge, an additional axiom holds that explicitly forbids the superposition of eigenstates of different values of the conserved quantity.  

In particular, Wick, Wightman, and Wigner define a super-selection rule as a restriction on all measurable quantities, $\mathcal{O}$, that is Hermitian operators, such that the expectation value of these observables for states in a superposition of different values of a conserved quantity is zero.  Whereas in standard quantum mechanics all Hermitian operators correspond to observables and vice versa~\cite{vN55}, super-selection rules restrict the set of observables to be a strict subset of all Hermitian operators.  Consequently, Wight, Wightman, and Wigner introduce a dichotomy between 
some conserved quantities like momentum, angular momentum, and energy, for which no super-selection rules apply,  and conserved quantities like parity and charge for which super-selection rules do apply.  

This dichotomy between conserved quantities was challenged by Aharonov and Susskind a decade later~\cite{AS67}.  In particular, Aharonov and Susskind devised a thought experiment where coherent superpositions of different charge eigenstates could be prepared.  Their thought experiment was an adaptation of the Stern-Gerlach experiment, used to construct coherent superpositions of angular momentum eigenstates.  Moreover, Aharonov and Susskind showed that under suitable conditions a super-selection rule can exist even for angular momentum.  Their argument is as follows.  Consider an arrangement with  two Stern-Gerlach magnets and a fluorescent screen after the second magnet. The first magnet acts as a preparation devise, while the second as a measurement devise.  The first Stern-Gerlach magnet's orientation changes constantly and unpredictably while the second magnet is under the control of the experimenter.  An electron initially in a state of definite angular momentum (say ``up'' in $z$-axis according to some absolute fixed frame of reference) enters the first magnet, whose action corresponds to a rotation about an axis $\vec{n}\in\mathbb{R_3}$ by an amount $\theta\in[0,2\pi)$.  The state of the electron immediately after the first Stern-Gerlach magnet can be described as pointing `'up'' along  the axis $\vec{m}\in\mathbb{R_3}$.  Now suppose that the second Stern-Gerlach magnet is aligned along direction $\vec{m}$ and applies a rotation by an angle $\theta'\in[0,2\pi)$.  As the state of the electron is an eigenstate of this rotation, the state of the system immediately after exiting the second Stern-Gerlach magnet is still ``up'' along $\vec{m}$.  Suppose that there was some mechanism (such as a tracking device) such that the relative orientation between the first magnet, and the second magnet together with the fluorescent screen remains fixed.   Then after many electrons pass through this arrangement, a single bright spot will be seen on the screen, indicating coherence.

On the other hand suppose that the relative orientation between the first magnet and the second magnet and fluorescent screen does not remain fixed, but changes randomly and unpredictably (no tracking device).  Then the state of the first electron immediately after exiting the first Stern-Gerlach magnet is ``up'' along $\vec{m_1}$, the state of the second electron electron is ``up'' along $\vec{m_2}$.  In general the state of the $N^{\mathrm{th}}$ electron would be ``up'' along $\vec{m_N}$ where the directions $\vec{m_1},\ldots,\vec{m_N}$ are distributed randomly over the surface of a sphere.  Hence, the bright spots appearing on the fluorescent screen after the $N$ electrons pass through the second magnet will be randomly distributed along a circle, meaning that the system is in an incoherent state.  Hence, an observer whose measuring apparatus (the second-Stern-Gerlach magnet and fluorescent screen) is correlated with the first Stern-Gerlach magnet sees a coherent superposition of angular momentum eigenstates, whereas an observer whose second magnet is uncorrelated with the first does not.  The latter thus concludes that there is a super-selection rule in place.   

Hence, the ability to prepare and observe coherent superpositions of eigenstates of different angular momenta can be achieved with the help of an external system (in this case the pair of Stern-Gerlach magnets) that serves as a frame of reference for the isolated system in question.  Aharonov and Susskind proceed to construct such an external frame of reference relative to which a coherent superpostion between two different charge eigenstates can be prepared and observed.  The authors conclusion is that super-selection rules are not fundamental in nature, but arise in situations where the  requisite reference frame for the physical system in question is lacking.
   
%Moreover, Aharonov and Susskind showed that it is possible for a super-selection rule to hold even for angular momentum.  In particular, Aharonov and Susskind argued that, if a party prepares an isolated quantum system in a definite eigenstate of momentum, then that party can never observe a superposition of eigenstates of different angular momenta.  However, if a second party who possess a pair of Stern-Gerlach magnets, whose relative orientation is known, then superpositions of eigenstates of different angular momenta can be prepared and observed.  

In a reply to Aharonov and Susskind, Wick, Wightman, and Wigner~\cite{WWW70} state that no super-selection rule exists for position, linear momentum or angular momentum because there exist natural systems, such as particles, magnets etc, that are in effect coherent superpositions of eigenstates of the conserved quantities in question. For example the position of a system is measured relative to a stationary particle.  By the uncertainty principle, since the particle is localized in space, it's momentum, the conjugate variable to position, is indefinite.  That is it's quantum mechanical state is a coherent superposition of states of linear momentum.  Wick, Wightman, and Wigner argued that no natural systems exist that are in a coherent superposition of the physical quantity conjugate to charge, or parity.  Aharonov and Susskind's argument, as well as Mirman's who helped elucidate as to how such systems can be constructed~\cite{M69, M79}, is that one cannot elevate conservation laws to super-selection rules purely on the fact that no natural systems are known to act as appropriate reference frames.  In short, the only difference between conserved quantities like momentum, position and angular momentum, that do not satisfy super-selection rules, and of quantities like charge, baryon number, and parity, that do is the difficulty of preparing and maintaining the appropriate frames of reference.  It is worth noting that since the super-selection debate, several proposals for constructing coherent superpositions of charge eigenstates, involving superconductors, have been proposed~\cite{KW74}, as well as coherent superpositions of eigenstates of atom number in Bose-Einstein condensates~\cite{CD97, HY96,JY96,YRJ97,DBRS06}.

The connection between reference frames and super-selection rules has been addressed in more detail in the field of quantum information, and in particular quantum optics, where significant progress has been made, particularly in the last decade. As early as 1963,  Glauber had shown that the state of the output field of a laser can be described as a coherent superposition of photon number eigenstates~\cite{G63}. Subsequent quantum optical experiments showed that Glauber's prediction did indeed explain all quantum optical phenomena observed in the lab, the most important of which is interference, a trademark phenomenon attributed to quantum states exhibiting coherence between various eigenstate of photon number.  

In addition to the experimental results, the coherent state representation was easy to work with and calculate.  As a result the quantum optics community was convinced that the true state of the field outputted by a laser was indeed a coherent state.  In a seminal paper in 1997, M{\o}lmer challenged this notion, and showed that by a different line of argument one could reach the conclusion that what is outputted by a laser is not a coherent superposition of photon number eigenstates, but an incoherent mixture of photon number eigenstates,i.e.~that there is a photon number super-selection rule in place for the optical field of a laser~\cite{M97}.  Furthermore, M{\o}lmer showed, via numerical simulations, that this description of the laser is not at odds with experimental results arguing that two lasers satisfying the photon number super-selection rule can indeed interfere.  The result caused a stir in the quantum optics community and became known as the optical coherence controversy.

The conclusions of M{\o}lmer were analytically proven a few years later by Sanders, Bartlett, Rudolph and Knight~\cite{SBRK03}.  Using an operational approach, and carefully analysing what really goes on in a quantum optics experiment, Sanders, Bartlett, Rudolph, and Knight showed that indeed, all quantum optical observations can be explained equally well if one assumes that the laser field is subject to a photon number super-selection rule. The authors point out that the density matrix describing the state of the laser filed is a mixed state (a mixed state is a density matrix whose rank is strictly greater than one). A well known property for such states is the freedom of ensemble decompositions~\cite{NC00}; two different ensembles of pure states may be described by the same density matrix.  Indeed, the choice of writing down an ensemble for a given density matrix is a matter of convenience.  Sanders, Bartlett, Rudolph, and Knight show that while it is true that the state of the laser field can be described by an ensemble of coherent states, it is also true that it can be written as an ensemble of photon-number eigenstates.  The choice of which ensemble is used is a mathematical convenience. The authors then invoke operationalism to argue that all that is ever observed in a quantum optics experiment (by observed here we mean measured empirically) is photon numbers.  A quantum optical detector is in actuality a photon counting machine, and a laser a device that produces pulses of definite photon numbers, with an appropriate probability distribution over the number of photons.  Thus, empirically all quantum optical experiments satisfy a photon-number super-selection rule. 

Sanders, Bartlett, Rudolph, and Knight then proceed to analyze quantum optical experiments under the photon number super-selection rule.  They show that all experimental phenomena, including interference, homodyne measurements, and coherence of a multimode laser field (a laser that outputs photons of not just one frequency (mode) but of several frequencies) can still be adequately explained within the photon number super-selection framework.   I note in passing that similar debates arose also in the supeconductor community, where the debate is whether the state of a superconductor is actually a superposition of charge eigenstates~\cite{H62,KW74,A86}, or whether a charge superselection rule is in place, and in Bose-Einstein condensation, where the debate is whether the state of a Bose-Einestein condensate is in a coherent superposition of eigenstates of atom number, or whether there is an atom-number super-selection rule in place~\cite{JY96,HY96,YRJ97,CD97}.

The solution to the controversy came a few years later by Bartlett, Rudolph and Spekkens~\cite{BRS06}.  The controversy is resolved by noticing that quantum coherence is reference frame dependent.  The main idea in the authors line of argument is that the quantum state of a system contains information not only about the system itself, but also about the relation of the system to other systems external to it.  Thus, whether or not a system is in a coherent superposition depends on the external systems one uses to describe it.

This is very reminiscent of the paradoxes that arise in special relativity.  Consider two observers, Alice and Bob, who are in relative motion to each other and the well-known paradox of the ladder and the barn.  The barn and ladder are of length L, such that when the two are stationary the ladder fits exactly inside the barn.  Now Bob, who is in a rocket car capable of traveling close to the speed of light, has an identical copy of this ladder mounted on top of his vehicle which is moving towards the barn.  According to Bob, the barn is moving towards him at a speed close to the speed of light, so that it suffers length contraction.  Consequently, Bob infers that the ladder does not fit entirely inside the barn.  But according to Alice, who is in the same frame of reference as the barn, it is the ladder that is moving close to the speed of light.  Hence, the ladder suffers length contraction, and fits exactly inside the barn.  In fact Alice notices that there is non-zero time interval between the point in time where the back end of the ladder enters the barn, and the point in time where the front end of the ladder exits the barn.  

The resolution to the barn and ladder paradox is well-known.  Both observers are equally correct in their description of the phenomenon.  The paradox arises because the phenomenon is viewed from two different frames of reference.   What Bartlett, Rudolph and Spekkens claim is the resolution of the coherence controversy is exactly this.  The controversy arises because the state of the quantum system is viewed from two different frames of reference.  In particular, coherences are seen by an observer whose frame of reference is correlated with the system.  In a sense the coherence in the quantum state of a system contains information about the correlation of the system to an external frame of reference.  On the other hand, if the external frame of reference is uncorrelated with the system, then the quantum state of the latter will not, in general, be described by a coherent superposition. 

Bartlett, Rudolph, and Spekkens argue that a super-selection rule applies to an observer whose frame of reference is uncorrelated to that used to prepare the state of the quantum system.  On the other hand, if the two frames of references are correlated, then no super-selection rules apply and coherent superpositions can be prepared and measured.  Super-selection rules that arise due to a party lacking an appropriate frame of reference are referred to as \defn{induced} super-selection rules.  Super-selection rules such as photon number, linear or angular momentum, and position are induced super-selection rule.  On the other hand super-selection rules such as charge or atom number are referred to as \defn{axiomatic}.  The difference between induced and axiomatic super-selection rules is precisely the remark made by Wick, Wightman, and Wigner; induced super-selection rules are associated with conserved quantities for which reference frames exist naturally, but the experimenter does not have access to such systems, whereas axiomatic super-selection rules are those associated with conserved quantities for which no natural systems exist to act as frames of reference. 

From the above discussion, it is clear that  possessing a system that can be used as a reference frame relative to which other systems can be described alleviates an induced super-selection rule~\cite{BRS07}. Indeed, the authors state that there is no reason why axiomatic super-selection rules cannot be equally alleviated, other than the difficulty of preparing and maintaing an appropriate frame of reference.  Hence, a frame of reference for a particular conserved quantity is a \defn{resource}.  Having such a system allows one to prepare quantum systems in a superposition of eigenstates of the relevant conserved quantity.  A few questions then naturally arise. Does a party lacking a requisite frame of reference describe entanglement the same way as a party who does possess a frame of reference.  Do the two parties agree on the amount of entanglement they possess? What kind of restrictions are imposed, in terms of quantum information processing tasks, if a party lacks a requisite frame of reference, and how best to quantify the ability of a physical system to act as a frame of reference? These questions have given rise to a whole new field of investigation known as the resource theory of reference frames.  In the following we review some of the answers to the questions posed above.   

To better understand the nature of the restrictions imposed by the lack of a shared frame of reference we need to define what we mean by a frame of reference.  In the most abstract sense a frame of reference is the minimum number of coordinates needed to uniquely specify a point in space. The most familiar example of a reference frame is the Cartesian frame in three dimensions.  Any point in three dimensional space is uniquely specified by a triple of real numbers $(x_i,y_i,z_i)$ that denote the distance of the point from some reference point, say $(0,0,0)$.  The choice of this reference point is not unique.  For example, one might choose to describe the same point above relative to $(x',y',z')\in\mathbb{R}$, where $x', y', z'\neq0$, yielding the answer $(a_i,b_i,c_i)$, where $a_i, b_i, c_i$ denote the distance of the point from $(x',y',z')$.  The two discriptions are ofcourse related by a \defn{coordinate transformation}, a transformation that maps $(0,0,0)$ to $(x',y',z')$.  The later is known as a shift of the origin, or a translation.

The most trivial coordinate transformation is the identity transformation corresponding to doing nothing to the reference frame. Furthermore, if two coordinate transformations are composed together, that is enacted in succession, upon a frame of reference, then the resulting frame of reference is related to the intial one by a single coordinate transformation.  Thus, the set of coordinate transfromations is closed under composition. In addition coordinate transfromations satisfy the assosiative property.  Finally, for every coordinate transformation, there exists another coordinate transformation, called the inverse, which undoes it. It follows that the set of all possible coordinate transformations of a reference frame form a group.  In the example given above, the relevant group of transformations is the translation group in three dimensions.

Thus for every abstract reference frame there corresponds a particular group, $G$, that describes the set of all possible coordinate transformations of the reference frame. Groups are known to describe symmetries of objects.  For example the symmetries of a tetrahedron, all possible transformations of the tetrahedron whose result is indistinguishable from doing nothing to the tetrahedron, are described by the group of permutation of four objects, $S_4$.  In physics, symmetries of systems are known to give rise to conservation laws~\cite{N18}.  For example, consider two observers whose frames of reference are related by a translation.  A free particle is passing by both these observers. Whereas the two observers disagree as to the position of the particle, they both agree on the value of the velocity of the particle (and in fact in the value of the acceleration of the particle).  Hence, both observfers agree on the equations of motion describing the particle.  Thus, the linear momentum of systems is translation invariant, i.e.~for any translation of coordinates, the value of linear momentum of a system remains the same.  
      
Returning to quantum information, then, one might ask if entanglement is a property that remains invariant under the action of a group associated with a requisite reference frame. Entanglement is an important resource in quantum information.  If two parties are restricted to local operations and classical communication alone, then entanglement can be used in place of a quantum channel, obviating the restrictions.  Verstraete and Cirac discovered that in the presence of super-selection rules there the notion of non-locality has to be redifned.  There exists states that cannot be prepared locally and exhibit non-local properties~\cite{VC03}.  In addition, under a super-selection rule, it is no longer true that any two orthogonal states can be perfectly distinguished.  Verstraete and Cirac use such states to construct a data hiding protocol, which under the assumption of a local particle number super-selection rule, which is perfect provided the parties do not share any resource states.  In a data hiding protocol, classical or quantum information is distributed amongst several parties in such a way that the message can be read if and only if the parties are provided with the means to perform joint measurements.  It is known that for unrestricted quantum mechanics a perfect data hiding protocol is not possible~\cite{TDL01,DLT02}.  Non-locality of a single photon is another example where the notion of entanglement in the unrestricted quantum formalism (violation of Bell-inequality, ability to perform quantum teleportation and super-dense coding or the inability to prepare such a state via LOCC) no longer apply.  In particular, the state for which the debate of non-locality of a single photon arose is a state that does not satisfy the first two operational notions, but satisfies the third one.

Verstraete and Cirac also showed that any quantum information protocol that can be performed in a universe without super-selection rules can also be implemented in a world with super-selection rules.  However, as their data hiding example showed, it might be possible to be able to perform some tasks in a super-selected universe that are otherwise impossible to do in an unrestricted universe.  This was proven not to be the case by Kitaev, Mayers, and Preskill~\cite{KMP04}.  The authors showed that the information theoretic security (that is the unconditional security) of any quantum information protocol cannot be enhanced by the presence of super-selection rules.  To prove that a protocol is unconditionally secure, one must show that the protocol is secure against any cheater who possess unlimited resources.  Under this assumption, then, nothing forbids the cheater from possessing reference system that help lift the super-selection rule.  In particular, for the data hiding protocol proposed by Verstraete and Cirac, Kitaev, Mayers, and Preskill showed that a cheating party in such a protocol can have access to the non-local states described by Verstraete and Cirac, and obtain information about the hidden information.  More generally, the authors showed that an omnipotent party can always be assumed to possess an appropriate reference system that lifts the restriction of the super-selection rule.  Hence the information security of protocols subject to super-selection rules can be no better than those in the standard quantum formalism. 

The analysis of Verstraete and Cirac, as well as Kitaev, Mayers, Preskil shows that under super-selection rules traditional notions of entanglement, particularly the measures used to quantify entanglement, have to be revised.  in addition, under super-selection rules additional resources are present, such as states that can act a reference frames and alleviate the restriction.  Schuch, Verstraete, and Cirac quantified the latter resource by a single additive quantity which they called the super-selected induced variance~\cite{SVC04a, SVC04b}.  The authors showed that the non-local resources for a protocol subject to the restriction of local operations and classical communication (LOCC) as well as super-selection can be quantified by two additive quantities.  The former by the entropy of entanglement~\cite{BBPS96}, while the latter by the super-selected induced variance.  The latter quantifies the ability of a state of a quantum system to stand in the place of a frame of reference for the relevant super-selection rule, and explains the reason why some states while separable, cannot be prepared locally under a super-selection rule and can be used to obviate the restriction as described in the data hiding protocol above.  

The subject of properly quantifying entanglement under a super-selection rule had been considered earlier by Vollbrecht and Werner~\cite{VW01}.  In particular, the authors consider how to compute two measures of entanglement, the relative entropy of entanglement and the entanglement of formation~\cite{BDSW96}, for states that possess a particular symmetry, such as rotationally invariant states~\cite{W89}.  The importance of quantifying the entanglement under a particular symmetry is of practical use as well, particularly in certain quantum computing implementations such as NMR quantum computing.  In the latter quantum states are represented by different atoms on a molecule and operations are enacted using radio frequencies and an antennae.  If more than one molecule are present then operations act collectively on all the atoms.  This imposes a constraint on the type of operations that can be performed in such a setting.  All operations must be symmetric, i.e.~permutation invariant.  Hence a super-selection rule associated with the symmetric group, $S_N$. is in place.  

As NMR quantum computing is a very promising quantum computing technology, it is important not to over-estimate the amount of entanglement in such a scenario.  Bartlett, and Wiseman showed that under such restrictions the amount of entanglement is much less than what one would naively think by not taking into account the super-selection rule~\cite{BW03, WBV03}.  A similar analysis for a particle number super-selection rule was under-taken where it was shown that the amount of entanglement per particle is asymptotically zero under a particle number super-selection rule, a surprising result considering that in the absence of particle number super-selection, the amount of entanglement per particle is strictly non-zero~\cite{WV03,WBV03}.     

It was later shown that entanglement under LOCC and a super-selection rule is more analogous to the entanglement theory of mixed states under LOCC alone~\cite{BDSW06,JWBVP06}.  In particular, the property above where the entanglement per particle is zero under a particle number super-selection rule is analogous to bound entanglement present in mixed states~\cite{HHH98}.  The latter is a property of mixed states that cannot be locally prepared and whose entanglement is inaccessible.  Under a super-selection rule associated with a group $G$, the authors showed that there is a gap between the set of states that can be prepared under a $G$-SSR and the set of states that can are distillable under $G$-SSR.  The gap is made up of states that are similar to the bound entangled states in LOCC alone.  However, the authors showed that such entanglement can be activated if the parties are provided with a suitable reference state. i.e.~the entanglement is bound by the restrictions imposed by the super-selection rule and a suitable reference states lifts these restrictions activating the entanglement.  Hence, with a suitable reference state all states are either prepared locally or distillable.   superselection background
 

 


%In this thesis, I will focus on induced super-selection rules.  In particular i will focus on super-selection rules that are induced due to uncorrelated frames of reference, i.e.~between two or more parties whose corresponding frames of reference are uncorrelated.  Reference frames are related by coordinate transformations.  The set of all coordinate transformations that map a reference frame to itself form a group $G$. As reference frames are physical systems, this group corresponds to the symmetry group of the system acting as a reference frame.  For example, consider a frame of reference for position.  A physical system that can act as such a frame of reference is a particle relative to which the positions of all other particles are defined.  The symmetry group describing all coordinate transformations of our reference particle is the group of translations in three spatial coordinates.  Similarly, a classical gyroscope, defines a reference frame for spatial rotations.  The group of symmetries for such a reference frame is the special orthogonal group in three dimensions, the group of all three by three matrices with determinant one.  

      

          



 







\end{document}